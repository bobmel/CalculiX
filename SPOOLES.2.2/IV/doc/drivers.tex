\par
\section{Driver programs for the {\tt IV object}}
\label{section:IV:drivers}
\par
%=======================================================================
\begin{enumerate}
%-----------------------------------------------------------------------
\item
\begin{verbatim}
testIO msglvl msgFile inFile outFile
\end{verbatim}
This driver program tests the {\tt IV} IO methods,
and is useful for translating between the formatted {\tt *.ivf}
and binary {\tt *.ivb} files.
\par
\begin{itemize}
\item
The {\tt msglvl} parameter determines the amount of output.
Use {\tt msglvl = 1} for just timing output.
\item
The {\tt msgFile} parameter determines the message file --- if {\tt
msgFile} is {\tt stdout}, then the message file is {\it stdout},
otherwise a file is opened with {\it append} status to receive any
output data.
\item
The {\tt inFile} parameter is the name of the file 
from which to read in the object.
{\tt inFile} must be of the form {\tt *.ivf} for a formatted file
or {\tt *.ivb} for a binary file.
\item
The {\tt outFile} parameter is the name of the file 
to which to write out the object.
If {\tt outfile} is of the form {\tt *.ivf}, the object is written
to a formatted file.
If {\tt outfile} is of the form {\tt *.ivb}, the object is written
to a binary file.
When {\tt outFile} is {\it not} {\tt "none"}, 
the object is written to the file in a human readable format.
When {\tt outFile} is {\tt "none"}, the object is not written out.
\end{itemize}
%-----------------------------------------------------------------------
\end{enumerate}
